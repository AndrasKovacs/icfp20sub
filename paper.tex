%% For double-blind review submission, w/o CCS and ACM Reference (max submission space)
\documentclass[acmsmall,review,anonymous]{acmart}\settopmatter{printfolios=true,printccs=false,printacmref=false}
%% For double-blind review submission, w/ CCS and ACM Reference
%\documentclass[acmsmall,review,anonymous]{acmart}\settopmatter{printfolios=true}
%% For single-blind review submission, w/o CCS and ACM Reference (max submission space)
%\documentclass[acmsmall,review]{acmart}\settopmatter{printfolios=true,printccs=false,printacmref=false}
%% For single-blind review submission, w/ CCS and ACM Reference
%\documentclass[acmsmall,review]{acmart}\settopmatter{printfolios=true}
%% For final camera-ready submission, w/ required CCS and ACM Reference
%\documentclass[acmsmall]{acmart}\settopmatter{}


%% Journal information
%% Supplied to authors by publisher for camera-ready submission;
%% use defaults for review submission.
\acmJournal{PACMPL}
\acmVolume{1}
\acmNumber{CONF} % CONF = POPL or ICFP or OOPSLA
\acmArticle{1}
\acmYear{2018}
\acmMonth{1}
\acmDOI{} % \acmDOI{10.1145/nnnnnnn.nnnnnnn}
\startPage{1}

%% Copyright information
%% Supplied to authors (based on authors' rights management selection;
%% see authors.acm.org) by publisher for camera-ready submission;
%% use 'none' for review submission.
\setcopyright{none}
%\setcopyright{acmcopyright}
%\setcopyright{acmlicensed}
%\setcopyright{rightsretained}
%\copyrightyear{2018}           %% If different from \acmYear

%% Bibliography style
\bibliographystyle{ACM-Reference-Format}
%% Citation style
%% Note: author/year citations are required for papers published as an
%% issue of PACMPL.
\citestyle{acmauthoryear}   %% For author/year citations


%%%%%%%%%%%%%%%%%%%%%%%%%%%%%%%%%%%%%%%%%%%%%%%%%%%%%%%%%%%%%%%%%%%%%%
%% Note: Authors migrating a paper from PACMPL format to traditional
%% SIGPLAN proceedings format must update the '\documentclass' and
%% topmatter commands above; see 'acmart-sigplanproc-template.tex'.
%%%%%%%%%%%%%%%%%%%%%%%%%%%%%%%%%%%%%%%%%%%%%%%%%%%%%%%%%%%%%%%%%%%%%%


%% Some recommended packages.
\usepackage{booktabs}   %% For formal tables:
                        %% http://ctan.org/pkg/booktabs
\usepackage{subcaption} %% For complex figures with subfigures/subcaptions
                        %% http://ctan.org/pkg/subcaption


\begin{document}

%% Title information
\title{Elaboration with First-Class Implicit Function Types}


%% Author with single affiliation.
\author{Andr{\'a}s Kov{\'a}cs}
\orcid{0000-0002-6375-9781}
\affiliation{
  \department{Department of Programming Languages and Compilers}
  \institution{E{\"o}tv{\"o}s Lor{\'a}nd University}
  \city{Budapest}
  \country{Hungary}
}
\email{kovacsandras@inf.elte.hu}


\begin{abstract}
Implicit functions are dependently typed functions, such that arguments are
provided (by default) by inference machinery instead of programmers of the
surface language. Implicit functions in Agda are an archetypal example. In the
Haskell language as implemented by the Glasgow Haskell Compiler (GHC),
polymorphic types are another example. Implicit function types are
\emph{first-class} if they are treated as any other type in the surface
language. This is true in Agda and partially true in GHC. Inference and
elaboration in the presence of first-class implicit functions poses a challenge;
in the context of GHC and ML-like languages, this has been dubbed
``impredicative instantiation'' or ``impredicative inference''. We propose a new
framework for elaborating first-class implicit functions, which is applicable
for full dependent type theories and compares favorably to prior solutions in
terms of power, generality and conceptual simplicity. We build atop Norell's
bidirectional elaboration algorithm for Agda, and note that key issue is
incomplete information about insertions of implicit abstractions and
applications. We make it possible to track and refine information related to
such insertions, by adding a new function type to a core Martin-L\"of type
theory, which supports strict (definitional) currying. This allows us to
represent undetermined domain arities of implicit function types, and we can
decide at any point during elaboration whether implicit abstractions should be
inserted.
\end{abstract}


%% 2012 ACM Computing Classification System (CSS) concepts
%% Generate at 'http://dl.acm.org/ccs/ccs.cfm'.
\begin{CCSXML}
<ccs2012>
<concept>
<concept_id>10011007.10011006.10011008</concept_id>
<concept_desc>Software and its engineering~General programming languages</concept_desc>
<concept_significance>500</concept_significance>
</concept>
<concept>
<concept_id>10003456.10003457.10003521.10003525</concept_id>
<concept_desc>Social and professional topics~History of programming languages</concept_desc>
<concept_significance>300</concept_significance>
</concept>
</ccs2012>
\end{CCSXML}

\ccsdesc[500]{Software and its engineering~General programming languages}
\ccsdesc[300]{Social and professional topics~History of programming languages}
%% End of generated code

%% Keywords
%% comma separated list
\keywords{impredicative polymorphism, type theory, elaboration, type inference}
\maketitle


\section{Introduction}

\begin{itemize}
\item Elaboration.

\item Hole filling.

\item Insertion of implicits.

\item First-class implicit functions vs Coq.

\item Unique vs non-unique solutions.

\end{itemize}

\section{Bidirectional Elaboration}

\begin{itemize}
\item

\item Limitations in Agda

\item Insertion of implicits.

\item First-class implicit functions vs Coq.

\item Unique vs non-unique solutions.

\item On the uniqueness of implicit insertion.
\end{itemize}






\begin{acks}
  This work was supported by the European Union, co-financed by the
  European Social Fund (EFOP-3.6.3-VEKOP-16-2017-00002) and COST Action
  EUTypes CA15123.
\end{acks}


%% Bibliography
\bibliography{references}


\end{document}
